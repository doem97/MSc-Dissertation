%=== CHAPTER SIX (6) ===
%=== Conclusion and Recommendations ===

\chapter{Conclusion and Recommendations}
\label{cha:conclusion}
\begin{spacing}{1.5}
\setlength{\parskip}{0.3in}

\section{Conclusions}

The new trend of utilizing deep learning in the ADAS system has been driving the auto-driving area in recent years. Lane and road marking detection are one of the most critical tasks within this topic. There are two main problems in the lane and road marking detection: the curved road and the bad weather conditions. Traditional networks perform well on straight roads but can not predict the trend of curved roads. Besides, extreme weather conditions, like reflection, blurring, and low brightness are critical factors that confuse the prediction. In the previous, some research focus on the road marking detection, and some others focus on lane detection, while no research combined the lane and road marking tasks together. Some research utilizes vanishing points to predict the lane in a traditional mathematical way, but no significant deep-learning-based methods of using vanishing points are proposed.

In our work, we proposed a Refined VPGNet, the RVPG, which mainly has three features. First, it does the lane detection and road marking classification simultaneously. Secondly, it utilizes the vanishing point to guide the lane detection task. Thirdly, it creatively uses the 4-Map, 4-Tiling, and 2-D Gaussian layers to combine training information better.

The network uses multi-task to make multiple predictions at the same time. In the network structure, the multi-task branches bifurcate after a fully convolutional layers stack. By using a convolutional layers stack, it extracts the higher-level features from the input. Multi-task has four branches: the grid box task, the masking task, the multi-label classification task, and the vanishing point task. Using these four tasks, the data used by any task are shared among all four tasks; thus, the network obtains more information and can perform better.

The vanishing point branch gives additional information when predicting the curved road. Inspired by the fact that the human eye uses the vanishing point to predict lanes' trends implicitly, RVPG explicitly feeds the vanishing point into the training process.

However, there is an apparent weakness in the multi-task and in using the vanishing point. In multi-task, the more branches there are, the more computational resources needed. In the vanishing point utilization, the single frame vanishing point is an invalid feed scheme; thus, a more efficient feeding VP scheme is needed.

RVPG uses a 4-Tiling layer, 4-Map layer, and 2-D Gaussian layer to address above mentioned issues. 4-Tiling layer is a brand new way to combine feature maps while not increasing the computing resources needed. By using this layer in the multi-task branches, we can save the computational resources. 4-Map layer is to split the vanishing point information into $5$ channels, four of which represent four corners around the vanishing point. 2-D Gaussian is another method that gives contextual information of VP. The use of 4-Map and 2-D Gaussian will prevent the network from being trapped.

This work is implemented initially in CAFFE, and we did a considerable amount of work to transplant it to PyTorch. Further researchers can save much time using our PyTorch implementation.

The experimental results are significant. The proposed network can achieve as high as 97.19\% multi-class accuracy. We conducted experiments with different epochs and collected the experimental loss and accuracy for further reference. In bad weather conditions, the network can achieve as high as 99.73\% $F_1$ score, even the scene is blurred heavily by raindrops.

\section{Recommendation in Future Work}

% \subsection{Detection under Rainy Condition}

In our work, the PyTorch implementation is still not entirely constructed. The basic structure is developed and open sourced\footnote{\url{https://github.com/PrabhuSM16/VPGNet-Pytorch}} already; thus, researchers can implement it very quickly without compiling the CAFFE environment. However, the 2-D Gaussian scheme is only implemented in Numpy currently and is needed to be implemented in PyTorch Tensors. 

For further study, a better-designed loss function may be helpful. After implementing the 2-D Gaussian function, the vanishing point is fed to the training process via Gaussian parameters, but that is not enough. As the 2-D Gaussian labels cover an oval area in the ground truth, a loss function measuring vanishing point's similarity in each direction can be designed. The longer the distance between the predicted vanishing point and the ground truth is, the more loss network will receive. Besides, based on the fact that curved lane trending has more information distributed in the horizontal direction but less information in the vertical direction, the loss function should have higher weights in the horizontal direction.

In multi-task branches, the grid box task is not very significant. Further research may try to remove it.

In our implementation, the network is not pixel-to-pixel mapping but has a scale-reducing (original image is $(480 \times 640)$ but ground truth is resized to $(120 \times 160)$). This is for reducing the computational resources. However, in future studies, researchers may have idle computational resources to do training on the original size if the grid box task is removed. 

% \subsection{Improvements on Few Sample Road Markings}



%=== END OF CHAPTER SIX ===
\end{spacing}
\newpage
